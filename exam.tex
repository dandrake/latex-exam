% search for "begin{questions" to jump to...well, you know.
% search for CHANGEME for things that need to be changed (date, class)
\documentclass{exam}
\usepackage{amsmath}
\usepackage[margin=2cm]{geometry}
\usepackage[colorlinks=false]{hyperref}
\usepackage{microtype}
% \usepackage{siunitx}

\renewcommand{\d}[1]{\,\mathrm{d}#1}
\newcommand{\e}{\mathrm{e}}

% For "compressed" exam -- no vfills, pagebreaks, etc -- uncomment
% below. The actual value is unimportant; it only matters that it's
% defined. You can also call latex with something like
%
% pdflatex '\newcommand{\compressedformat}{x}\input{exam1.tex}'
%
%\newcommand{\compressedformat}{x}

%
% In the body of the exam, use \myvfill, \mynewpage, and \mycompress
% (the latter is intended to wrap tikzpicture environments, but works
% with anything.)
%
\newcommand{\mycompress}[1]{\ifdefined\compressedformat\relax\else#1\fi}
\newcommand{\myvfill}{\mycompress{\vfill}}
\newcommand{\mynewpage}{\mycompress{\newpage}}

%
% Change these appropriately.
%
\newcommand{\thedate}{CHANGEME DATE}
\newcommand{\theexam}{CHANGEME EXAM}
\newcommand{\thecourse}{CHANGEME COURSE}

%
% to print a big "DRAFT" on each page, uncomment below
%
%\newcommand{\draftversion}{x}
\newcommand{\drafttext}{\ifdefined\draftversion
  \begin{tikzpicture}[remember picture,overlay]
  \node [rotate=60,scale=12,color=gray!25] at (current page.center)
  {\textsf{DRAFT}};
\end{tikzpicture}
\else
  \relax
\fi}

%
% use this to add "problem continues..."
%
\newcommand{\problemcontinues}{
  \ifdefined\compressedformat
    \relax
  \else
    \runningfooter{}{\thepage\ of \numpages}{problem continues\ldots}
    \newpage
    \runningfooter{}{\thepage\ of \numpages}{}
  \fi}

\begin{document}

\pagestyle{headandfoot}
\header{\thecourse}{\theexam}{\thedate\drafttext}
\runningheadrule
\firstpageheader{\drafttext}{}{}
\firstpagefooter{}{}{}
\runningfooter{}{\thepage\ of \numpages}{}

\addpoints
\parindent 0ex

\textbf{\thecourse} \hfill  \textbf{Name:}
\makebox[6cm]{\hrulefill}

\textbf{\theexam}

\textbf{\thedate} \hfill \textbf{ID number:}
\makebox[6cm]{\hrulefill}

\rule[1ex]{\textwidth}{.1pt}

\ifdefined\compressedformat

{\large Usual exam instructions would go here, if not doing compressed
  format.}

\else

This exam contains $\numpages$ pages (including this cover page) and
$\numquestions$ problems for a total of $\numpoints$ points. Check to
see if any pages are missing.\\

You may \emph{not} use your books, notes, a calculator, or any
electronic device. You have three hours to complete this exam.\\

You are required to show your work on each problem on this exam. The
following rules apply:

FIXME: Maybe something like, except for the follow-the-instructions bit:

Your job on this exam is to, as accurately and effectively as possible,
communicate your understanding (or perhaps lack thereof) of the
concepts and ability to use the skills required by these problems. Your
score on this exam will reflect how well you have communicated that
understanding and demonstrated those skills. In
particular:

\begin{itemize}

\item \textbf{Show your work,} in a logical, coherent, and reasonably
  neat way, in the space provided. Work scattered all over the page
  without a clear ordering will receive very little credit.

\item \textbf{Mysterious or unsupported answers will not receive full
    credit.} Your work should be mathematically correct and carefully
  and legibly written.

\item \textbf{A correct answer, unsupported by calculations,
    explanation, or algebraic work will receive no credit;} an incorrect
  answer supported by substantially correct calculations and
  explanations might still receive partial credit.

\item \textbf{Follow the instructions of the exam proctors.}

\item If you need more space, use the back of the pages; clearly
  indicate when you have done this.

\end{itemize}

\fi
\myvfill

\begin{center} \gradetable[h][questions] \end{center}

\mynewpage

\begin{questions}

\noaddpoints \question[15] \addpoints
Let $f(x) = \displaystyle\frac{x^3 -15x}{x^2 - 5}$.
\begin{parts}
  \part[5] Find all critical points of $f$.
  \myvfill
  \part[5] Find all intervals on which $f$ is concave down.
  \myvfill
  \part[5] Find all intervals on which $f$ is increasing.
  \myvfill
\end{parts}

\end{questions}

\end{document}

%%% Local Variables:
%%% mode: latex
%%% TeX-master: t
%%% End:
